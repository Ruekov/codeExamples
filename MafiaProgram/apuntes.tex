\documentclass[a4paper]{letter}
\usepackage[T1]{fontenc}
\usepackage[utf8]{inputenc}
\usepackage{lmodern}
\usepackage[spanish]{babel}

\usepackage{hyperref}

\hypersetup{
    pdfauthor={Guillem Rueda Cebollero},
    pdftitle={Apuntes y consideraciones de MafiaProgram}
}



\begin{document}
 \pagestyle{empty} 

Apuntes y consideraciones:\\

\begin{itemize}
\item Cuándo un miembro entra o sale de la prisión no es eliminado de la jerarquía, sólo el valor de actividad de su clase pasará a ser Cierta o Falsa.
\item Se guarda el estado del mafioso en su clase (\emph{mafiaMember}). No se deja de pertenecer a la jerarquía por ir a la cárcel, sólo se deja de actuar en ella.
\item La función \emph{highestMember} dados dos objetos \emph{mafiaMember} devuelve el que tenga el rango superior dentro la jerarquía.
\item En \emph{main.py} se prueba las distintas clases. No se hace un test de stress. En caso de stress, y disponibilidad de espacio hará falta analizar que consultas son más frecuentes y actuar en consecuencia.\\ 
\end{itemize}

Acerca la clase \textbf{member}:\\

\begin{itemize}
\item La clase es abstracta para poder dar cabida a posibles tipos de miembros distintos (policía, políticos,infiltrados, etc.).
\item Por ahora el valor clave es el nombre del miembro. Debería ser elegido otro valor, un valor único (número de ficha criminal, por ejemplo).
\item Pueden ser atorgados otros atributos a la classe. Sólo se han otorgado los imprescindibles (nombre y actividad).\\
\end{itemize}

Acerca la clase \textbf{organization}:\\

\begin{itemize}
\item La clase es virtual porqué tampoco sabemos que implementación tendrá (cada organización es distinta).
\item En el caso de mafia, se trata de una cola dónde los miembros veteranos tienen más poder que los nuevos.
\item Los veteranos seimpre estarán con los números más cercanos al indice 0.
\item Si un miembro fallece, se borrará de la jerarquía.\\
\end{itemize}

\end{document}
